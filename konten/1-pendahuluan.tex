\section{PENDAHULUAN}

\subsection{Latar Belakang}

% Ubah paragraf-paragraf berikut sesuai dengan latar belakang dari tugas akhir
%Saat ini era Industry telah memasuki generasi keempat pada revolusi industri atau lebih dikenal dengan revolusi industri 4.0 yang di mana pada babak ini mensinergikan aspek fisik dengan digital atau biasa disebut dengan digitalisasi. Pemanfaatan babak keempat ini dapat dilihat dari adanya pemanfaatan kecerdasan buatan (artificial intelligence), robotika, dan kemampuan komputer belajar dari data (machine learning). Machine learning merupakan bagian dari AI (artificial intelligence) yang menggunakan statistic, dimana dengan metode ini memungkinkan mesin (komputer) untuk mengambil keputusan berdasarkan data. Algoritma machine learning dirancang agar dapat belajar dan kemampuannya meningkat seiring waktu ketika terdapat data baru tanpa diprogram secara eksplisit \parencite{Bukusakti}. Dengan menggunakan machine learning maka dapat mendigitalisasikan citra yang diambil dari webcam dan nantinya akan diambil sebagai data untuk diolah oleh komputer. Salah satu implementasi yaitu menangkap gestur tubuh. Gestur kumpulan dari pose yang digunakan untuk komunikasi non verbal dengan sikap yang dibuat tubuh atau gerakan dari tangan, wajah, atau anggota lain dari tubuh yang terlihat mengkomunikasikan pesan-pesan tertentu \parencite{gesturtangan}. Menggunkan teknologi machine learning maka gestur dari tubuh akan dapat diterjemahkan ke dalam logika pemrograman dengan begitu maka gestur tubuh ini nantinya akan dapat di implementasikan dalam banyak hal seperti membantu membenarkan pose berolahraga, menerjemahkan bahasa isyarat, dan menjadi sistem kendali pada robot. Perkembangan bidang robotika saat ini berkembang secara pesat, awalnya robot hanya dapat dikendalikan secara dekat, namun beberapa tahun berikutnya robot sudah bisa dikendalikan dengan jarak yang jauh dengan tanpa kabel atau wireless dan dikendalikan dengan remote control. Teknologi kendali robot telah dikembangkan yang dapat langsung bergerak sesuai dengan inputan dari gestur manusia. Terdapat penilitian tentang sistem kendali untuk mobil robot, namun penelitian ini menggunakan sensor accelerometer MPU-6050 yang diletakkan pada sarung tangan dan akan dipakai saat ingin mengendalikan mobil robot Arduino. Penelitian ini didapatkan tingkat akurasi 100\% dan hasil respon sensor terhadap mobil mampu

Pose adalah potongan-potongan dari gerakan yang betujuan untuk komunikasi non verbal yang digunakan untuk menyampaikan pesan penting, baik secara langsung maupun tidak langsung tanpa menggunakan kata-kata. Setiap pose dapat memiliki arti tersendiri sesuai dengan kesepakatan umum ataupun personal yang melakukan komunikasi. Pose tangan berarti suatu sikap yang diberikan oleh tangan untuk berkomunikasi \parencite{gesturtangan}. \par
Pose tangan ini biasa digunakan oleh teman-teman tuli untuk berkomunikasi, namun dengan perkembangan teknologi dimana adanya teknologi \textit{human computer interaction} yang memungkinkan untuk berkomunikasi dengan komputer menggunakan pose tangan. \textit{Human Computer Interacttion} atau biasa disingkat HCI adalah suatu bidang studi yang merancanga teknologi komputer untuk lebih interaktif saat dipakai. Adanya HCI ini dapat memudahkan manusia untuk menyelesaikan tugas-tugasnya dengan bantuan komputer. HCI memungkinkan manusia untuk berinterkasi lebih dengan benda-benda disekitanya, salah satu contohnya adalah penggunaan gestur tangan pada saat menggunakan \textit{handphone}, mematikan atau menyalakan lampu dengan perintah suara, serta menggerakkan robot \parencite{HCI}. \par
Pada saat ini perkembangan HCI cukup pesat yang diakibatkan dari perkembangan pada teknologi \textit{machine learning} dan robotika. Teknologi \textit{machine learning} memungkinkan komputer untuk belajar dari data-data yang diberikan oleh manusia dan juga mengambil keputusan dari data-data tersebut. Algoritma machine learning dirancang agar dapat belajar dan kemampuannya meningkat seiring waktu ketika terdapat data baru tanpa diprogram secara eksplisit. Pada salah satu cabang \textit{machine learning} yaitu \textit{computer vision} memungkinkan untuk komputer mengerti gerakan atau pose yang diberikan manusia melalui gambar atau vidio \parencite{Bukusakti}. Tidak hanya \textit{machine learning} yang berkembang namun pada robotika juga berkembang terutama pada sistem kendalinya. Awalnya robot hanya dapat dikendalikan secara dekat, namun beberapa tahun berikutnya robot sudah bisa dikendalikan dengan jarak yang jauh dengan tanpa kabel atau wireless dan dikendalikan dengan remote control. Teknologi kendali robot telah dikembangkan yang dapat bergerak sesuai dengan inputan dari gestur manusia melalui bantuan sensor-sensor yang diletakkan pada tangan atau bagian tubuh lainnya dan juga dapat menggunakan \textit{computer vision} untuk mengolah gestur tubuh manusia.
 


\subsection{Rumusan Masalah}

% Ubah paragraf berikut sesuai dengan rumusan masalah dari tugas akhir
Berdasarkan latar belakang di atas, pada Tugas Akhir ini diajukan rancangan mendeteksi gestur tangan dengan menggunakan Mediapipe dan diimplementasikan untuk mengendalikan mobil robot Arduino. Langkah pertama yaitu mengumpulkan dataset melalui ekstraksi citra dari webcam menggunakan Mediapipe. Kemudian dilakukan proses learning menggunakan machine learning serta pengujian akurasi deteski gestur tangan. Setelah itu data pembacaan gestur tangan tersebut akan dikirimkan kepada receiver yang ada pada mobil dan akan diterjemahkan menjadi input perintah logika pemrograman untuk menentukan aksi gerakan robot.

\subsection{Batasan Masalah atau Ruang Lingkup}

Dalam pembuatan Tugas Akhir ini, pembahasan masalah dibatasi beberapa hal berikut :
\begin{enumerate}
	\item Pengambilan citra akan dilakukan dengan menggunakan webcam yang terdapat pada Laptop.
	\item Pendeteksian tangan akan menggunakan \textit{framework mediapipe}.
	\item Pembatasan pose tangan yang menjadi dataset hanya akan ada 5 yakni maju, mundur, berhenti, belok kiri, dan belok kanan.
	\item Pengujian yang dilakukan adalah pengujian akurasi klasifikasi pada pose tangan.
\end{enumerate}

\subsection{Tujuan}

% Ubah paragraf berikut sesuai dengan tujuan penelitian dari tugas akhir
Berdasarkan latar belakang dan rumusan masalah diatas, mendapatkan tujuan pada Tugas Akhir ini sebaga berikut :
\begin{enumerate}
	\item Mendeteksi tangan menggunakan Mediapipe
	\item Mengetahui gestur tangan menggunakan Convolutional Neural Network
	\item Membuat sistem kendali untuk mobil robot arduino menggunakan gestur tangan
\end{enumerate}

\subsection{Manfaat}

% Ubah paragraf berikut sesuai dengan tujuan penelitian dari tugas akhir
Manfaat dari penelitian ini adalah \lipsum[8][1-14]

\section{PENDAHULUAN}

\subsection{Latar Belakang}

% Ubah paragraf-paragraf berikut sesuai dengan latar belakang dari tugas akhir
Saat ini era Industry telah memasuki generasi keempat pada revolusi industri atau lebih dikenal dengan revolusi industri 4.0 yang di mana pada babak ini mensinergikan aspek fisik dengan digital atau biasa disebut dengan digitalisasi. Pemanfaatan babak keempat ini dapat dilihat dari adanya pemanfaatan kecerdasan buatan (artificial intelligence), robotika, dan kemampuan komputer belajar dari data (machine learning). Machine learning merupakan bagian dari AI (artificial intelligence) yang menggunakan statistic, dimana dengan metode ini memungkinkan mesin (komputer) untuk mengambil keputusan berdasarkan data. Algoritma machine learning dirancang agar dapat belajar dan kemampuannya meningkat seiring waktu ketika terdapat data baru tanpa diprogram secara eksplisit (Hidayatullah, 2021). Dengan menggunakan machine learning maka dapat mendigitalisasikan citra yang diambil dari webcam dan nantinya akan diambil sebagai data untuk diolah oleh komputer. Salah satu implementasi yaitu menangkap gestur tubuh. Gestur merupakan komunikasi non verbal dengan sikap yang dibuat tubuh atau gerakan dari tangan, wajah, atau anggota lain dari tubuh yang terlihat mengkomunikasikan pesan-pesan tertentu (Kendon, 2004). Menggunkan teknologi machine learning maka gestur dari tubuh akan dapat diterjemahkan ke dalam logika pemrograman dengan begitu maka gestur tubuh ini nantinya akan dapat di implementasikan dalam banyak hal seperti membantu membenarkan pose berolahraga, menerjemahkan bahasa isyarat, dan menjadi sistem kendali pada robot. Perkembangan bidang robotika saat ini berkembang secara pesat, awalnya robot hanya dapat dikendalikan secara dekat, namun beberapa tahun berikutnya robot sudah bisa dikendalikan dengan jarak yang jauh dengan tanpa kabel atau wireless dan dikendalikan dengan remote control. Teknologi kendali robot telah dikembangkan yang dapat langsung bergerak sesuai dengan inputan dari gestur manusia. Terdapat penilitian tentang sistem kendali untuk mobil robot, namun penelitian ini menggunakan sensor accelerometer MPU-6050 yang diletakkan pada sarung tangan dan akan dipakai saat ingin mengendalikan mobil robot Arduino. Penelitian ini didapatkan tingkat akurasi 100\% dan hasil respon sensor terhadap mobil mampu

\subsection{Rumusan Masalah}

% Ubah paragraf berikut sesuai dengan rumusan masalah dari tugas akhir
Berdasarkan latar belakang di atas, pada Tugas Akhir ini diajukan rancangan mendeteksi gestur tangan dengan menggunakan Mediapipe dan diimplementasikan untuk mengendalikan mobil robot Arduino. Langkah pertama yaitu mengumpulkan dataset melalui ekstraksi citra dari webcam menggunakan Mediapipe. Kemudian dilakukan proses learning menggunakan machine learning serta pengujian akurasi deteski gestur tangan. Setelah itu data pembacaan gestur tangan tersebut akan dikirimkan kepada receiver yang ada pada mobil dan akan diterjemahkan menjadi input perintah logika pemrograman untuk menentukan aksi gerakan robot.

\subsection{Batasan Masalah atau Ruang Lingkup}

Dalam pembuatan Tugas Akhir ini, pembahasan masalah dibatasi beberapa hal berikut :
\begin{enumerate}
	\item Pendeteksian gestur tangan akan menggunakan metode mediapipe
	\item Pembatasan gestur tangan yang menjadi dataset hanya akan ada 5 yakni maju, mundur, berhenti, belok kiri, dan belok kanan.
	\item Pengujian yang dilakukan adalah pengujian akurasi deteksi dan respon mobil robot.
	\item Pengujian akan dilakukan dengan menggunakan webcam yang terhubung dengan Laptop.
\end{enumerate}

\subsection{Tujuan}

% Ubah paragraf berikut sesuai dengan tujuan penelitian dari tugas akhir
Berdasarkan latar belakang dan rumusan masalah diatas, mendapatkan tujuan pada Tugas Akhir ini sebaga berikut :
\begin{enumerate}
	\item Mendeteksi tangan menggunakan Mediapipe
	\item Mengetahui gestur tangan menggunakan Convolutional Neural Network
	\item Membuat sistem kendali untuk mobil robot arduino menggunakan gestur tangan
\end{enumerate}

\subsection{Manfaat}

% Ubah paragraf berikut sesuai dengan tujuan penelitian dari tugas akhir
Manfaat dari penelitian ini adalah \lipsum[8][1-14]

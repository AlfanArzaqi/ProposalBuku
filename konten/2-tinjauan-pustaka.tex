\section{TINJAUAN PUSTAKA}

% Ubah konten-konten berikut sesuai dengan isi dari tinjauan pustaka
\subsection{Hasil penelitian/perancangan terdahulu}
\subsubsection{Kendali mobil robot menggunakan isyarat tangan berbasis arduino}
Pada November 2020, Jati Widyo Leksono, Agung Samudra, Nanndo Yannuansa, dan Ahmad Fauzi membuat jurnal mengenai sistem kendali mobil robot menggunakan isyarat tangan. Dalam penelitian ini akan dilakukan pengembangan mobil robot yang mampu berjalan sesuai dengan isyarat tangan atau gestur tangan yang diberikan oleh user. Pada telapak tangan user nantinya akan ditempelkan suatu sensor yang mampu membaca pergerakan telapak tangan menurun, naik, kiri, kanan, dan mendatar. Dari pembacaan sensor tersebut data akan dikirimkan menuju receiver yang terdapat pada robot secara wireless. \parencite{JurnalElectroLuecat}

\subsubsection{}

\subsubsection{Convolutional Neural Network (CNN)}


% Contoh penggunaan referensi dari pustaka
Newton pernah merumuskan \parencite{Newton1687} bahwa \lipsum[8]
% Contoh penggunaan referensi dari persamaan
Kemudian menjadi persamaan seperti pada persamaan \ref{eq:FirstLaw}.

% Contoh pembuatan persamaan
\begin{equation}
  % Label referensi dari persamaan yang dibuat
  \label{eq:FirstLaw}
  % Baris kode persamaan yang dibuat
  \sum \mathbf{F} = 0\; \Leftrightarrow\; \frac{\mathrm{d} \mathbf{v} }{\mathrm{d}t} = 0.
\end{equation}

\lipsum[9]

\subsubsection{Anti Gravitasi}

\lipsum[10]

\begin{center}
  \large
  \textbf{KENDALI \textit{MOBILE ROBOT} BERBASIS POSE TANGAN MENGGUNAKAN \textit{CONVOLUTIONAL NEURAL NETWORK} (CNN)}
\end{center}
\addcontentsline{toc}{chapter}{ABSTRAK}
% Menyembunyikan nomor halaman
\thispagestyle{empty}

\begin{flushleft}
  \setlength{\tabcolsep}{0pt}
  \bfseries
  \begin{tabular}{ll@{\hspace{6pt}}l}
  Nama Mahasiswa / NRP&:& Alfan Miftah Arzaqi / 0721 19 4000 0003\\
  Departemen&:& Teknik Komputer FTEIC - ITS\\
  Dosen Pembimbing&:& 1. Ahmad Zaini, S.T., M.Sc.\\
  & & Dr. Eko Mulyanto Yuniarno,S.T.,M.T.\\
  \end{tabular}
  \vspace{4ex}
\end{flushleft}
\textbf{Abstrak}

% Isi Abstrak
Pose tangan biasa digunakan oleh teman-teman tuli untuk berkomunikasi, namun dengan perkembangan teknologi \textit{Human Computer Interaction} atau HCI yang memungkinkan untuk berkomunikasi dengan komputer menggunakan pose tangan. Pose tangan yang akan digunakan disini ada 5 pose yang akan menyimbolkan maju, mundur, berhenti, belok kanan, dan belok kiri. Proses kalsifikasi akan menggunakan CNN dalam melakukan training dan akan menggunakan citra yang sudah disimpan dari tahapan dataset. Dari data yang telah di klasifikasikan dan sudah didapatkan posenya maka pose tersebut akan diterjemahkan ke dalam suatu perintah untuk dapat menggerakkan \textit{mobile robot}. Robot akan menerima data dari hasil klasifikasi dan selanjutnya akan bergerak sesuai data yang diberikan.

\vspace{2ex}
\noindent
\textbf{Kata Kunci: CNN, \emph{Mobile Robot}, Pose,}